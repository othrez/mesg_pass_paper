\section{Implementation}
\label{sec:optim}

The tool is available on GitHub \cite{github.MPass}.
It includes the sources files, protocols specifications in \xml\ and user manual.
%
\MPass\ tool is implemented in \texttt{C++} with the help of \texttt{lemon} and \texttt{pugixml} libraries.

\subsubsection{Optimisations.}

Various optimization techniques were implemented in order to increase the efficiency of the approach described in  \cite{AAC13}:

\begin{description}
\item{\textbf{Reducing the number of copies per process:}}
The original approach consisted in making $k$ copies of each process from the protocol specification.
Instead of that, we make only two copies per process (send and receive copy).
%
\item{\textbf{Removal of strongly connected component:}}
We evaluate all the sets of strongly connected components in the send copy of each process. 
Then, we replace each strongly connected component by two new states.
We add a send operation between these two newly added states if this operation appears in this set of strongly connected component.
The first of these two new states will be the the entering point, or state, for any transition inbounding this strongly connected component.
The other state (final state) will be  the source state for all transitions leaving from the previous strongly connected component.
Thus, the number of states per process is reduced, which, in consequence, reduces the \texttt{smt}-formula to be checked.
\end{description}
