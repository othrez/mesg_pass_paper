\section{Implementation}
\label{sec:optim}

The tool is available on GitHub \cite{github.MPass}.
It includes sources files, protocols specifications in \Xml\ and the user manual.
%
\MPass\ tool is implemented in \texttt{C++} with the help of \texttt{lemon} and \texttt{pugixml} libraries.
%
%In order to use \MPass, \textsc{z3} is also required.

\subsubsection*{From the \Xml\ protocol specification to Automata representation.}
\label{subsec:copies}
Protocols are specified in \Xml\ files in the \texttt{Includes} sub-folder of the tool repository.
%
They can be modified in a simple manner by adding, modifying or removing the \Xml\ rules and
new protocols can be specified by writing their specification into an \Xml\ file.
%
The first task of \MPass\ is to translate the protocol into a set of Non-Deterministic Finite Automaton (\Nfa).
%
It does so by parsing the \Xml\ file which path is given as input in the \texttt{settings} file (using the \texttt{plugixml} library).
%
Then, it makes use of the \texttt{lemon} library to translate the protocol into a set of \Nfa,
each \Nfa\ defining one process from the protocol.
%
For each such generated \Nfa, \MPass\ proceeds by extracting two automata,
one containing all except the receive transitions (\textit{send copy} of that process),
the other one containing all except the send transitions (\textit{receive copy} of that process).
%
In total, \MPass\ would have extracted $2*N$ automata from the input protocol
($N$ being the number of processes composing the protocol).

\subsubsection*{From Automata to Quantifier-free Presburger Formulas.}
The reachability problem of a given protocol state is analyzed by generating a \emph{quantifier-free Presburger formula}
from the set of extracted send an receive automata.
%
The \Smt\ solver \texttt{z3} is then fed with the generated formula to check its satisfiability.
%
If the \Smt-solver proves the quantifier free Presburger formula to be satisfiable,
then the state is reachable and we have an unsafe condition.
Otherwise, the formula is unsatisfiable, the state is unreachable and the program is safe.
%
Note that the result of the reachability analysis should be interpreted with regard to the phase bound and with regard to the channel semantic
under which the analysis has been carried.

More information regarding the translation of the bounded phase state reachability to the satisfiability of \emph{quantifier-free Presburger formula} can be found in \cite{AAC13}, but here are some details concerning this translation.
%
In order to generate the quantifier-free Presburger formula,
a certain number of variables need to be defined in order to encode states and transitions of every \Nfa.
%
This includes the \texttt{index}, the \texttt{occurrence} and the \texttt{match} variables.
%
Variables \texttt{index} are integer variables used in order to index all states of all automata,
while \texttt{occurrence} are boolean variables that apply to transitions and tell if the transition has been taken or not with respect to a given run.
%
Variables \texttt{match} apply to receive transitions and check whether the receive transition matches a \textit{previous} send operation.
%
The order of visited states and fired transitions  along a run is encoded by ordering the indices (\texttt{index} variables) of the states.


\subsubsection*{Optimisations.}
Various optimizations were implemented in order to increase the efficiency of the approach described in  \cite{AAC13} and implemented in \alternator\ \cite{github.alternator}:

\begin{description}
\item{\textbf{Reduction of the number of copies per process:}}
Since the reachability analysis is carried under a certain phase bound $k$, the original approach consisted in making $k$ copies of each process from the protocol specification.
Instead of that, we make only two copies per process (send and receive copies).
\MPass\ then generates variables for both send and receive copies of each process and duplicates them \emph{k} times which are then further used to generate the Presburger formula.
%
Thus, by duplicating variables instead of processes, we have saved space and time during the Presburger formula generation phase.
%we have ignored the inefficient process of making $2*k$ copies for each process as it is described in \cite{AAC13} and as it was implemented in \texttt{alternator} \cite{github.alternator}.
%
\item{\textbf{Removal of strongly connected components:}}
We evaluate all the strongly connected components (\Scc) of the send copy of each process. 
Then, we replace each \Scc\ by two new states.
We add a send operation between these two newly added states if this operation appears in the set of \Scc\ transitions.
The first of these two new states will be the entering point (state) for any transition inbounding this \Scc.
The other state will be  the source state for all transitions leaving the previous \Scc.
%
Thus, the number of states and transitions per process is reduced, which, in consequence, reduces the number of generated variables and assertions in the \Smt-formula.
\item{\textbf{Reduction of the number of assertions:}} The \texttt{z3} single command \texttt{create\_distinct\_statement} was used in order make all state indices distinct,
thus replacing all assertions of the form \texttt{index(state$_{\texttt{1}}$) != index(state$_{\texttt{2}}$)}.
This considerably reduced the number of assertions fed to the \Smt\ solver.
\end{description}
