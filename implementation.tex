\section{Implementation}
\label{sec:optim}

The tool is available on GitHub \cite{github.MPass}.
It includes the sources files, protocols specifications in \xml\ and user manual.
%
\MPass\ tool is implemented in \texttt{C++} with the help of \texttt{lemon} and \texttt{pugixml} libraries.
%
%In order to use \MPass, \textsc{z3} is also required.

\subsubsection*{From the \Xml\ protocol specification to Automata representation.}
\label{subsec:copies}
Protocols are specified in \Xml\ files in the \texttt{Includes} sub-folder of the tool repository.
%
They can be modified in a simple manner by adding, modifying or removing the \Xml\ rules and
new protocols can be specified by writing their specification into an \Xml\ file.
%
The first task of \MPass\ is to translate the protocol into a set of Non-Deterministic Finite Automaton (\Nfa).
%
It does so by parsing the \Xml\ file which path is given as input in the \texttt{settings} file (using the \texttt{plugixml} library).
%
Then, it makes use of the \texttt{lemon} library to translate the protocol into a set of \Nfa,
each one defining one process from the protocol.
%
For each such generated \Nfa, \MPass\ proceeds by extracting two automata,
one containing all except the receive transitions (\textit{send copy} of that process),
the other containing all except the send transitions (\textit{receive copy} of that process).
%
In total, \MPass\ would have extracted $2*N$ automata from the input protocol
($N$ being the number of processes composing the protocol).

\subsubsection*{From Automata to Quantifier-free Presburger Formulas.}
The reachability problem of a given protocol state is analyzed by generating a \emph{quantifier-free Presburger formula}
from the set of extracted send an receive automata.
The \Smt\ solver \texttt{z3} is then fed with the generated formula to check its satisfiability.
%
In order to generate the quantifier-free Presburger formula, a certain number variables have to be 
defined for each transition of each extracted automata.
This includes the \texttt{index}, the \texttt{occurrence} and the \texttt{match} variables.
%
Since the reachability analysis is carried under a certain phase bound $k$,
\MPass\ generates variables for both send and receive copies of each process and then duplicates them \emph{k} times
which are then further used to generate Presburger formulas.
Thus, we have ignored the inefficient process of making multiple copies for each process as done in \cite{AAC13}.
%
More information regarding the translation of the bounded phase state reachability to the satisfiability of \emph{quantifier-free Presburger formula} can be found in \cite{AAC13}.

If the \Smt-solver proves the the quantifier free Presburger formula to be satisfiable,
then the state is reachable and we have an unsafe condition.
Otherwise, if the formula is unsatisfiable, then the state is unreachable for the given bound and the program is safe.

%\begin{figure}[h]
%\begin{center}
%\begin{tabular}{l@{\hspace{20pt}}}
%$
%\left(\occvarof{\transition}=1\right)
%\wedge
%\left(\occvarof{\transition'}=1\right)
%\wedge
%\left(\indexvarof{\transition}<\indexvarof{\transition'}\right)
%\implies
%\left(\matchingvarof{\transition}<\matchingvarof{\transition'}\right)
%$.
%\end{tabular}
%\end{center}
%\caption{An example of a \emph{quantifier-free Presburger formulas}}\label{fig:examples}
%\end{figure}

\subsubsection*{Optimisations.}
Various optimizations were implemented in order to increase the efficiency of the approach described in  \cite{AAC13}:

\begin{description}
\item{\textbf{Reduction of the number of copies per process:}}
If $k$ is the phase bound, the original approach consisted in making $k$ copies of each process from the protocol specification.
Instead of that, we make only two copies per process (send and receive copies).
%
\item{\textbf{Removal of strongly connected components:}}
We evaluate all the sets of strongly connected components in the send copy of each process. 
Then, we replace each strongly connected component by two new states.
We add a send operation between these two newly added states if this operation appears in the set of strongly connected component transitions.
The first of these two new states will be the entering point, or state, for any transition inbounding this strongly connected component.
The other state (final state) will be  the source state for all transitions leaving from the previous strongly connected component.
Thus, the number of states per process is reduced, which, in consequence, reduces the number of variables in the \Smt-formula.
\end{description}
