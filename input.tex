\section{Input format}
\label{sec:input}

Examples on which we run our protocols are mostly taken from \cite{JRSVgit} 
which are further described in \cite{MPSV11} and \cite{RSV11}.
Bounded Retransmission Protocol protocol is also adapted from \cite{AABJ04}.
All protocols used for our experiments can be found in the \texttt{Includes/Protocols} folder.
%
In \cite{JRSVgit} and in \texttt{alternator}, the authors used a tabular format to specify a protocol (\texttt{.csv} file)
from which an \xml\ description could be extracted.
Our tool works directly on the \xml\ description.
%
More specifically, \MPass\ takes as input a \texttt{settings} file where users can feed the tool with a structured input containing the following informations:

\begin{itemize}
\item \textsc{xml\_file}: Path to \xml\ file where the protocol is specified.
\item \textsc{channel\_semantic}: Either \texttt{lossy}, \texttt{stuttering} or \texttt{unordered} semantics.
\item \textsc{phase bound}: How many phases are allowed per process.
\item \textsc{bad state}: It takes pairs of values.
%
First is the name of the process in which the reachability is to be checked and the second is the set of states for which reachability is being checked.
NOTE: Bad state for more one process can also be entered one after the other.
\end{itemize}

In order to ask \MPass\ to check the $3$-bounded reachability of state \texttt{Invalid} of process \texttt{RECEIVER} of the \texttt{BRP} protocol
using the \texttt{lossy} semantic,
you can feed the tool with the following settings file, which you can find in the \texttt{src} directory of \MPass\ :\\
\texttt{file ../Includes/Protocols/BRP.xml}\\
\texttt{semantics lossy}\\
\texttt{bound 3}\\
\texttt{bad RECEIVER Invalid}\\
\texttt{channel\_type process}

\subsection*{Protocol specification}

The \texttt{xml} specification of all protocols used for our experiments can be found in the \texttt{Includes/Protocols} folder.
%
Variants of these protocols can be specified by modifying their \xml\ specification
and new protocols can be specified by writing their \xml\ specification.
%
An \texttt{xml} protocol description contains:
\begin{inparaenum}
\item The name of the protocol. Example: \texttt{Altenating Bit Protocol}.
\item The communicating channel semantic. Example: \texttt{STUTT\_FIFO} for the stuttering semantic.
\item The set of messages exchanged between processes.
\item The set of channels.
\item A list of process specifications. The specification of a process contains:
\begin{inparaenum}
\item The name of the process.
\item The set of states of the process.
\item The set of transitions, each one specified as a "rule". An example of a process transition specification is depicted in figure~\ref{fig:xml:trans}.
\end{inparaenum}
\end{inparaenum}

\begin{figure}[h]
\begin{center}
\begin{tabular}{l@{\hspace{20pt}}}
$<${\bf rule} id="Q0{\textunderscore\textunderscore}ack1{\textunderscore\textunderscore}INBOUND"{\bf $>$}\\
\quad {\bf $<$pre$>$}\\
    \quad \quad {\bf $<$current{\textunderscore}state$>$}Q0{\bf $<$/current{\textunderscore}state$>$}\\
    \quad \quad {\bf $<$received{\textunderscore}message$>$}ack1{\bf $<$/received{\textunderscore}message$>$}\\
    \quad \quad {\bf $<$channel$>$}c1{\bf $<$/channel$>$}\\
  \quad {\bf $<$/pre$>$}\\
  \quad {\bf $<$post$>$}\\
    \quad \quad {\bf $<$next{\textunderscore}state$>$}Q1{\bf $<$/next{\textunderscore}state$>$}\\
     \quad \quad {\bf $<$send{\textunderscore}message$>$}mesg0{\bf $<$/send{\textunderscore}message$>$}\\
      \quad \quad {\bf $<$channel$>$}c0{\bf $<$/channel$>$}\\
  \quad {\bf $<$/post$>$}\\
{\bf $<$/rule$>$}\\
\end{tabular}
\end{center}
\caption{Example of a transition description (\textsc{ABP} Protocol).}
\label{fig:xml:trans}
\end{figure}

The above rule specifies two transitions from  state {\tt Q0} to  state {\tt Q1}.
The first transition consists in receiving message {\tt ack1} from channel {\tt c1}.
The second transition consists in sending message {\tt mesg0} to the channel {\tt c0}.
The set of transition rules, together with the set of states define a process.
The set of processes, together with channel and message definitions, define a protocol.


