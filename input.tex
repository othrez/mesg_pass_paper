\section{Input format}\label{sec:input}

Examples on which we run our protocols are mostly take from \cite{JRSVgit} 
which are further described in \cite{MPSV11} and \cite{RSV11}.
Bounded Retransmission Protocol is also adapted from \cite{AABJ04}.

In ?-?, the authors used a tabular format to specify a protocol (\trextsc{.csv} file)
from which an \texsc{xml} description of the protocol could be extracted.
Our tool works directly on \textsc{xml} files.
More specifically, \MPass\ takes as input a \textsc{settings} file where users can feed the tool with a structured input containing the following informations:

\begin{itemize}
\item \textsc{xml_file}: Path to \textsc{xml} file where the protocol is specified.
\item \textsc{channel_semantic}: Either ... .
\item \textsc{phase bound}: How many phases are allowed per process.
\item \textsc{bad state}: Which state of which process is considered as bad state, i.e.
which state of which process should the state reachability analysis be conducted for.
\end{itemise}

An example of a \texsc{Settings}, which can be found in the \texsc{src} directory is the following:
\textsc{TEXT FILE}

Protocols are used in the format of {\tt xml-files} present within the tool repository inside the 'Includes' folder. The protocols, thus, can be modified in a simple manner by changing the fields in xml-files. The first task of MPass is to translate the protocols defined in {\tt xml-files} into 
\emph{Non-Deterministic Finite Automata} (NFA). In order to achieve this, it takes xml-file path of the Protocol 
as an input and then uses C++ library of {\tt lemon} to translate the protocol into NFA as described below:

\begin{figure}[h]
\begin{center}
\begin{tabular}{l@{\hspace{20pt}}}
$<${\bf rule} id="Q0{\textunderscore\textunderscore}ack1{\textunderscore\textunderscore}INBOUND"{\bf $>$}\\
\quad {\bf $<$pre$>$}\\
    \quad \quad {\bf $<$current{\textunderscore}state$>$}Q0{\bf $<$/current{\textunderscore}state$>$}\\
    \quad \quad {\bf $<$received{\textunderscore}message$>$}ack1{\bf $<$/received{\textunderscore}message$>$}\\
    \quad \quad {\bf $<$channel$>$}c1{\bf $<$/channel$>$}\\
  \quad {\bf $<$/pre$>$}\\
  \quad {\bf $<$post$>$}\\
    \quad \quad {\bf $<$next{\textunderscore}state$>$}Q1{\bf $<$/next{\textunderscore}state$>$}\\
     \quad \quad {\bf $<$send{\textunderscore}message$>$}mesg0{\bf $<$/send{\textunderscore}message$>$}\\
      \quad \quad {\bf $<$channel$>$}c0{\bf $<$/channel$>$}\\
  \quad {\bf $<$/post$>$}\\
{\bf $<$/rule$>$}\\
\end{tabular}
\end{center}
\caption{An example of xml code for ABP Protocol}\label{fig:code}
\end{figure}

The above rule adds two transitions from the state {\tt Q0} to the state {\tt Q1}.
The first transition defines the rule of receiving the message {\tt ack1} from the channel {\tt c1} 
whereas the second transition defines the rule of sending the message {\tt mesg0} into the channel {\tt c0}.
Each process in a protocol contains one or more of  such rules which in together defines the automaton for that 
process within the given protocol.

If the result of the SMT-solver for these set of formulas is satisfiable (sat), then the \emph{Bad State} is reachable and we have an UNSAFE (U) condition otherwise we have a SAFE (S) condition, ie, 
\emph{Bad State} is  	 reachable for the given bound.


