\section{Input format}
\label{sec:input}

Examples on which we run our protocols are mostly taken from \cite{JRSVgit} 
which are further described in \cite{MPSV11} and \cite{RSV11}.
Bounded Retransmission Protocol (\Brp) is also adapted from \cite{AABJ04}.
All protocols used for our experiments can be found in the \texttt{Includes/Protocols} folder.
%
In \cite{JRSVgit} and in \alternator, the authors used a tabular format to specify a protocol (\texttt{.csv} file)
from which an \Xml\ description could be extracted.
Our tool works directly on the \Xml\ description.
%
More specifically, \MPass\ takes as input a \texttt{settings} file where users can feed the tool with a structured input containing the following informations:

\begin{description}
\item[\Xml\ file:] Path to the \Xml\ file where the protocol is specified.
\item[Channel semantic:] Either \texttt{lossy} (\Lcs), \texttt{stuttering} (\Slcs) or \texttt{unordered} (\Ucs) channel semantic.
\item[Phase bound:] How many phases are allowed per process.
\item[Bad state:] It takes pairs of values.
%
First is the name of the process,  second is the (bad) state (of that process) for which state reachability is to be checked.
%
Note that several bad states can be entered.
\end{description}

For instance, in order to ask \MPass\ to check the $3$-phase-bounded reachability
of the state \texttt{Invalid} of the process \texttt{RECEIVER} of the \Brp\ protocol
using the \texttt{lossy} semantic,
the user can feed the tool with the following settings file, which you can find in the \texttt{src} directory of \MPass:

\begin{figure}
\subcaptionbox{Example of a \texttt{settings} file.\label{fig:settings:file}}[0.4\linewidth]{
	\begin{tabular}{ c l c}
	&\texttt{\footnotesize file ../Includes/Protocols/}&\\
	&\texttt{BRP.xml}&\\
	&\texttt{\footnotesize semantics lossy}&\\
	&\texttt{\footnotesize bound 3}&\\
	&\texttt{\footnotesize bad RECEIVER Invalid}&\\
	&\texttt{\footnotesize channel\_type process}&\\
	\end{tabular}
}
%
\subcaptionbox{{\Xml\ representation of process transition (of the \Abp\ protocol)}\label{fig:xml:trans}}[0.58\linewidth]{
\begin{tabular}{l@{\hspace{20pt}}}
{\footnotesize $<${\bf rule} id=\texttt{"Q0{\textunderscore\textunderscore}ack1{\textunderscore\textunderscore}INBOUND"}{\bf $>$}}\\
{\footnotesize \quad {\bf $<$pre$>$}}\\
{\footnotesize    \quad \quad {\bf $<$current{\textunderscore}state$>$}{\tt Q0}{\bf $<$/current{\textunderscore}state$>$}}\\
{\footnotesize    \quad \quad {\bf $<$received{\textunderscore}message$>$}{\tt ack1}{\bf $<$/received{\textunderscore}message$>$}}\\
{\footnotesize    \quad \quad {\bf $<$channel$>$}{\tt c1}{\bf $<$/channel$>$}}\\
{\footnotesize  \quad {\bf $<$/pre$>$}}\\
{\footnotesize  \quad {\bf $<$post$>$}}\\
{\footnotesize    \quad \quad {\bf $<$next{\textunderscore}state$>$}{\tt Q1}{\bf $<$/next{\textunderscore}state$>$}}\\
{\footnotesize     \quad \quad {\bf $<$send{\textunderscore}message$>$}{\tt mesg0}{\bf $<$/send{\textunderscore}message$>$}}\\
{\footnotesize      \quad \quad {\bf $<$channel$>$}{\tt c0}{\bf $<$/channel$>$}}\\
{\footnotesize  \quad {\bf $<$/post$>$}}\\
{\footnotesize {\bf $<$/rule$>$}}\\
\end{tabular}
}
\caption{Input format}\label{fig:input}
\end{figure}


\subsubsection*{Protocol specification.}

The \Xml\ specification of all protocols used for our experiments can be found in the \texttt{Includes/Protocols} folder.
%
Variants of these protocols can be specified by modifying their corresponding \Xml\ specification.
Also, new protocols can be specified by writing a new \xml\ specification.
%
An \Xml\ protocol description contains:
\begin{inparaenum}
\item The name of the protocol. Example: \texttt{Altenating Bit Protocol}.
%\item The communicating channel semantic. Example: \texttt{STUTT\_FIFO} for the stuttering semantic.
\item The set of messages exchanged between processes.
\item The set of channels.
\item A list of process specifications. The specification of each process contains:
\begin{inparaenum}
\item The name of the process,
\item the set of states of the process and
\item the set of transitions, each transition specified as an \Xml\ ``rule''. An example of a process transition specification is depicted in figure~\ref{fig:xml:trans}.
\end{inparaenum}
\end{inparaenum}

The above rule specifies two transitions from  state {\tt Q0} to  state {\tt Q1}.
The first transition consists in receiving message {\tt ack1} from channel {\tt c1}.
The second transition consists in sending message {\tt mesg0} to the channel {\tt c0}.
The set of transition rules, together with the set of states define a process.
The set of processes, together with channel and message definitions, defines a protocol.


