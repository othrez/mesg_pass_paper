\section{Introduction}

\MPass\ is a tool dedicated to the verification of state reachability for message-passing programs 
communicating in an asynchronous manner via unbounded channels.
%
The reachability problem is undecidable in the case where the channels are \textit{perfect},
i.e. channels that behave as non-lossy queues\cite{BZ83} .
%
Because of the unbounded size of the channels, this message passing programs can be in fact considered as infinite state systems,
even if the number of the states of the program itself is finite.
%
Nevertheless, the problem becomes decidable if we consider other than perfect channel kind of semantics for the communication medium,
namely \textit{lossy}, \textit{stuttering} or \textit{unordered} channel semantics \cite{AB93,Rack78,phs-IPL2002,lipton}.
\textit{Lossy} semantic allows loss of messages from any channel at any possible transition of the program.
\textit{Stuttering} on the other hand allows any message in any channel to be replicated any number of times.
Finally, no order of arrival is kept in the \textit{unordered} semantic, 
i.e. the channel is a multiset of messages where only the number of each message is kept.
%
It turns out that the state reachability of message passing programs for these semantics
still comes with a high complexity.
%
Therefor, different approaches have been suggested in order to reduce the complexity of the reachability problem.
%
Most of these approaches consists in some sort of under approximation analysis of the system,
allowing the system to evolve up to a certain fixed bound, thus limiting the state space that has to be explored
in order to check the reachability of a given program state.
%
One of these approaches consists in bounding the number of \textit{context switches} between processes \cite{SJ05}.
%
Since many bugs get exposed after a small number of context switches,
it appears that these kind of approaches are very helpful when it comes to find bugs.
%
On the other hand, they can tell about the program safety only up to a certain bound, the one under which the analysis has been carried.
%
More specific to message passing communicating programs,
La Torre et al. considered in \cite{LaTorre08} a bounded context switch analysis for perfect channel communicating processes,
where processes are allowed to receive from only one channel in each context.


Parosh et al. considered in \cite{AAC13} a different approach, called \textit{bounded phase} analysis.
%
A \textit{phase} of the program is a run in which each process 
is either performing send operations or receive operations, but not both.
%
The number of send (resp. receive) operations within a sending (resp. receiving) phase
is not bounded, and the number of channels to which processes are sending to (resp. receiving from)
is only bounded by the number of channels composing the system.
%
Also, this approach does not put any restriction on the length of the
run or on the size of the buffers.
%
Finally, it allows multiple processes to be running at the same time.
%.
This approach has been demonstrated in a prototype, \textsc{alternator} \cite{github.alternator},
which proceeds in two steps in order to analyse the program.
%
First, it translates the bounded phase reachability problem of the program into the satisfiability problem of quantifier-free Presburger formula.
Second, it feeds the generated formula to an {\sc smt}-solver.
%
This allows leveraging the full power of state-of-the-art
{\sc smt}-solvers for obtaining a very efficient solution to the bounded-phase reachability problem for all above mentioned semantics. 
%
%[Global phase vs Local phase]
%
The promising results of this prototype motivated its reimplementation into a more efficient tool, \MPass\, that we present in this paper.

\MPass\ turns the bounded phase state reachability verification task into a push-button exercise.
Moreover, it comes with the following features:

\begin{itemize}
% input
\item Users of \MPass\ can feed the tool with a protocol specification in \textsc{xml}, the target state for which the reachability analysis should be conducted, the semantic of the channel and the phase bound. Details of the input format are provided in section~\ref{section:input}.
%
\item \MPass\ can handle three different kinds of channel semantics: {\it lossy}(LCS), {\it stuttering}(SLCS) or {\it multiset-unordered}(UCS) semantics which allow the messages inside the channels to be, respectively, lost, duplicated or re-ordered.
% internal
\item \MPass\ allows local (per process) bounded phase analysis.
This analysis explores more behaviours than the global (per program) bounded phase analysis implemented in \textsc{alternator}.
%
\item Several optimisations have been implemented in \MPass\, especially in the \textsc{smt}-formula generation phase (see section~\ref{section:optim}).
%
\item \MPass\ uses the \textsc{z3} {\sc smt}-solver \cite{z3}.
% output qualitative
\item If \MPass\ finds a bug then it is a real bug of the input program (under the considered channel semantics).
%
\item If the bound on the number of phases increases then the set of explored program behaviours increases.
In the limit, every run of the programs can then be explored.
% output quantitative
\item Users are provided with the computational time needed both by the constraint generation engine and by
  the \textsc{smt}-solver in order to analyse the reachability of the provided protocol and state.
%
\item \MPass\ output is generated in a user friendly manner, by generating a tex file.
More information regarding this can be found within the tool \cite{github.MPass}.
%
\item Finally, \MPass\ is an open source tool which can be interfaced with other \textsc{smt}-solvers.
\end{itemize}


\subsubsection{Targeted users}
%
\MPass\ can be used by the following group of users :
%
\begin{enumerate}
%
\item \textbf{Researchers} and \textbf{computer scientists} can use the freely available and open source code of \MPass\ to:
\begin{enumerate}
\item Compare the bounded phase approach with other approaches for the verification of message-passing programs,
\item Improve and optimize the implemented techniques,
either by generating a different \textsc{SMT}-formula or by using a different \textsc{SMT}-solver.
\item Target new platforms and programs (e.g. by adding shared variables or considering other channel semantics).
\end{enumerate}
%  
\item \textbf{Teachers} of distributed  systems  and concurrent programming classes can use (and augment) \MPass\ 
  to get their students accustomed with message-passing programs and protocols.
  In particular, the precision of \MPass\ can concretely illustrates the difficulty of writing a correct distributed protocol or algorithm.
%
\item \textbf{Software designers} that are working on message-passing programs can use \MPass\ to evaluate the safety of their tentative solutions.
  It can also be used to check faultiness of an already used protocol by verifying the reachability of an Invalid state.
%
\end{enumerate}